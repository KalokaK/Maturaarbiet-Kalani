% !TEX root = ../maturaarbeit.tex
\chapter{Material und Methode}\label{chap:material und methode}
Darlegung des Methodenansatzes.

\section{Über Einstein}\label{sec:ueber einstein}
Die mathematischen Methoden bestehen aus folgenden Termen:
\begin{equation}
	\label{eq:einstein}
	E = mc^2
\end{equation}
und
\begin{align}
	\label{eq:bernstein}
	F &= nd^2\\
	\label{eq:keinstein}
	\frac{G}{o} &= e^2
\end{align}
In \fref{eq:einstein} kann man ein Quadrat ($c^2$) erkennen. Die Herren Bernstein\footnote{Bernstein ist ein Halbedelstein.} und Keinstein haben mit \fref{eq:bernstein} und \fref{eq:keinstein} nichts neues zur Welt beigetragen.

\subsection{Dieser lange Titel sieht im Inhaltsverzeichnis unschön aus}\label{sec:fancyref1}
So kürzt man den im Inhaltsverzeichnis ab:\\
\verb+\subsection[kurz]{Ganz langer Titel, der nicht gut passt}+
\subsection[Automatik von fancyref]{Automatik von fancyref lässt sich abschalten}\label{sec:fancyref}
Und wenn uns das automatische \emph{Gleichung} von \verb+\fref+ nervt verwenden wir stattdessen Formel~(\ref{eq:bernstein}) mittels \verb+\ref+.
\newpage