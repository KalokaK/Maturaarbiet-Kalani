% !TEX program = pdflatex+makeindex+bibtex
% !TEX encoding = UTF-8 Unicodede_ch

% created by Simon Schälli <simon.schaelli@kantiwattwil.ch>
% v1.0 ursprüngliche Version
% v1.1 2015-10-22 mit Titelbild
% v1.2 2016-09-30 aufgeräumt
% v1.3 2016-10-16 Layoutfehler behoben

%======================================================================%
%=== Dokument auf häufige Fehler überprüfen                         ===%
%======================================================================%
\RequirePackage[l2tabu, orthodox]{nag}

%======================================================================%
%=== Dokumentklasse wählen                                          ===%
%======================================================================%
\documentclass[a4paper,openright]{scrreprt}

%======================================================================%
%=== Pakete einbinden                                               ===%
%======================================================================%
\usepackage[utf8]{inputenc}
\usepackage[english]{babel}

% Latin Modern Schriftart
\usepackage{lmodern}
\usepackage[T1]{fontenc}

% Mathematik-Pakete
\usepackage{amsmath}
\usepackage{amssymb}
\usepackage[mathscr]{euscript}
\usepackage{mathtools}

% mathematische Buchstaben als Text
\usepackage{textcomp}

% komfortable Referenzen mit \fref
\usepackage[english]{fancyref}

% Kopf- und Fusszeilen
\usepackage[automark,headsepline]{scrpage2}
\ihead[]{\headmark}
\chead[]{}
\ohead[]{\pagemark}
\ifoot[]{}
\cfoot[\pagemark]{}
\ofoot[]{}
\pagestyle{scrheadings}

% Einfügen von Bildern¨
\usepackage{float}
\usepackage{graphicx}
\usepackage{subfig}

% schöne Tabellen
\usepackage{booktabs}
\captionsetup[table]{font=small,skip=0pt}

% Tabelleninhalt an Dezimalpunkt ausrichten
\usepackage{dcolumn}
\makeatletter                     % <- hebt Sonderbedeutung von @ auf
\newcolumntype{d}[1]{D{.}{.}{#1}} % <- definiert einen neuen Spaltentyp
\makeatother                      % <- setzt Sonderbedeutung von @ wieder

% Algorithms
\usepackage[ruled,vlined]{algorithm2e}
\SetKwInput{KwRequire}{Require}

% code
\usepackage{xcolor}
\definecolor{bluekeywords}{rgb}{0.13,0.13,1}
\definecolor{greencomments}{rgb}{0,0.5,0}
\definecolor{redstrings}{rgb}{0.9,0,0}
\definecolor{codegreen}{rgb}{0,0.6,0}
\definecolor{codegray}{rgb}{0.5,0.5,0.5}
\definecolor{codepurple}{rgb}{0.5,0,1.0}

\usepackage{listings}
\lstset{language=Python,
showspaces=false,
showtabs=false,
breaklines=true,
showstringspaces=false,
breakatwhitespace=true,
escapeinside={(*@}{@*)},
keywordstyle=\color{bluekeywords}\bfseries,
stringstyle=\color{redstrings},
basicstyle=\ttfamily
}

%Color Boxes, proofs
\usepackage{tikz,lipsum,lmodern}
\usepackage[most]{tcolorbox}

% Beispieltext
\usepackage{lipsum}

% Hyperlinks in PDF
\usepackage{hyperref}

% Biblatex
\usepackage[
backend=biber,
style=numeric,
sorting=ynt
]{biblatex}
\addbibresource{mybib.bib}

% Macros
\usepackage{mymacros}
\usepackage{xargs}

%======================================================================%
%=== Angaben zum Dokument                                           ===%
%======================================================================%

\titlehead{Kantonsschule im Lee \hfill Kalani Fin Kistler\\
Fachschaft Mathematik \hfill Dättnauerstrasse 107\\
Rychenbergstrasse \hfill 8406 Winterthur\\
8400 Winterthur}
\subject{Matura Thesis 20/21}
\title{Policy Gradient Based Reinforcement Learning Applied to Playing Soccer}
\author{\texorpdfstring{Kalani Fin Kistler, 4h \\[1cm]\includegraphics[scale=0.2]{figures/Screenshot from 2020-09-22 23-05-59.png}\\[1cm] {\small Supervisor: Elena Fattorini}}{Kalani Fin Kistler 4h}}
\date{\small 04.01.2021}
%======================================================================%
%=== PDF Dokumenteinstellungen                                      ===%
%======================================================================%

\makeatletter
\hypersetup{
	pdftitle={\@title},%
	pdfsubject={\@subject},%
	pdfauthor={\@author},%
	pdfkeywords={},%
	colorlinks,%
	citecolor=black,%
	filecolor=black,%
	linkcolor=black,%
	urlcolor=black}
\makeatother
\usepackage[bottom=80pt]{geometry}


%======================================================================%
%=== Beginn des eigentlichen Dokumentes                             ===%
%======================================================================%

\begin{document}

\maketitle % <- Titel setzen
\cleardoublepage
\pagenumbering{roman} % <- römische Seitennummerierung
\tableofcontents % <- Inhaltsverzeichnis
\cleardoublepage % <- neue Seite
\pagenumbering{arabic} % <- arabische Seitennummerierung

% Kapitel einbinden:

\chapter*{NOTES FOR CHAPTER HAND IN ONLY, NOT ON FINAL}
Due to the fact that I am not yet familiar with vector graphics and wanted to focus on writing, the figures are temporary stand ins, meant only to convey the content of the final image, and are not the final images themselves.

% !TEX root = ../maturaarbeit.tex
\chapter{Introduction}\label{chap:einleitung}

% !TEX root = ../maturaarbeit.tex
\chapter{Background}\label{chap:background}
% !TEX root = ../maturaarbeit.tex
\chapter{Core concepts of Reinforcement Learning}\label{chap:theory}
\section{What is Reinforcement Learning?}\label{sec:RL}
\noindent
Throughout our daily lives we navigate our surroundings, handle social situations and tackle complex tasks. In doing, so we take the actions we believe, based on experience and intuition, to have the best outcome. We might take these abilities for granted, given how natural they are to us. However, at some point we had to obtain these skills so essential in managing our day to day, many through simple trial and error. Reinforcement Learning (RL) seeks to formalize the process of figuring out how to behave based on what produces desirable results and what does not, and adjusting our future actions accordingly. 

\noindent
\\ Reinforcement Learning is a discipline of machine learning \cite[p. 1]{sutton_reinforcement_2018}, an incredibly broad field which is focused on the optimization of computer algorithms by processing data and experience. As such it is an intersection between the natural, to us intuitive concept of learning, and the rigid and numerical world of computer programming and mathematics \cite[p. 4]{sutton_reinforcement_2018}. To be able to understand how this learning process works, and to be able to quantify and formalize it, it is essential to introduce some general concepts. 

\subsection{Illustrative Example: Card game UNO}\label{subsec:UNO}

\begin{figure}[ht]
    \centering
    \includegraphics[width=\linewidth]{figures/UNO.png}
    \caption{UNO environment in the context of RL}
    \label{fig:UNO}
\end{figure}

\noindent
Consider the example of teaching someone how to play the popular card game “UNO”. The \textit{objective} of the game is to get rid of the cards on your hand before the other players. We can call the game an \textit{environment}. This environment contains all the players players, including their strategies, the cards and what rules the game follows. A state of an environment describes the arrangement of all components encompassed by it, in this case the hands of the players, who's turn it is, what cards are on which pile and their order. Notice that the environment contains a lot of information which the player, in Reinforcement Learning called an \textit{agent} (an actor in the environment) does not know. However, the actor can \textit{observe} the environment and thereby gain a reasonably accurate representation of it’s state. Suppose it is the agent's turn. Based on its \textit{observation} the agent can take an \textit{action a}, playing any of the cards on its hand, or picking one up. The rules of an environment might forbid certain actions, taking one of them, will lead to the same environment state, where it is the agents turn. A well designed environment will give the agent negative feedback, when teaching someone UNO, that might be telling them they did something wrong. In Reinforcement Learning this is called \textit{reward signal}. Based on the \textit{reward}, the agent then can update its behaviour to produce a different action next time. Thereby the reward \textit{reinforces} the desired behaviour, it is a \textit{reinforcement} the agent learns from. The way of acting of an agent in a state given, an observation of that state, is called a \textit{policy}, commonly denoted as $\pi$. The agent follows a policy to decide on an action, based on a state.
\newline
For the sake of brevity, unless the distinction is relevant, I am going to use the terms observation and state interchangeably. Much more expansive definitions going in to the nuances of these concepts can be found in the book \textit{Reinforcement Learning, An introduction} by Richard Sutton and Andrew Barto. \cite{sutton_reinforcement_2018}

\subsection{What Type of Problem does Reinforcement Learning suit?}\label{subsec:Why_RL}
When solving a problem where the environments rules are well known like UNO Reinforcement Learning might indeed not be the best approach. In the case of UNO, it might be more efficient to design an algorithm which computes some probability of victory for each action, based on what cards have been played and then picks the optimal one. However, this approach requires knowledge of how the environment operates \cite[p. 8]{sutton_reinforcement_2018}, if for example, it was not know what happens after our turn, it would suddenly become nigh impossible to explicitly define a probability of victory for an action. Another issue that quickly arises as the environment becomes more complex, is the rapid rise in the number of its possible permutations. Accounting for all of them quickly becomes unfeasible as the environment grows more complex. Reinforcement Learning lets us generate high quality solutions in uncertain environments based on a reward signal and the goal it ultimately describes \cite[p. 03]{sutton_reinforcement_2018}. Another approach which might come to mind as an obvious solution would be to simply mimic the behaviour of an optimal, or close to optimal, agent. However, this again is impractical as such an agent might simply not exist or generating sufficient examples can be tedious. As stated in Reinforcement Learning, An Introduction: “In uncharted territory—where one would expect learning to be most beneficial, an agent must be able to learn from its own experience.” \cite[p. 02]{sutton_reinforcement_2018} Here it is important to keep in mind that machines see environments differently from humans. For a human the actions involved in pouring a coffee, might be to pick up the can and tilt it. Plenty of examples exist on how to do that. However, we are only able to make sense of the example because we intuitively understand how move our limbs, the instructions of "pick up and tilt" would be useless to a robot which views the world through a grid of pixels and which can take only take actions in the form of powering a series of motors which move an arm, it would need an example which matches its way of interaction with the environment.

\section{The Finite Markov Decision Processes, a mathematical framework for Reinforcement Learning}\label{sec:MDP} % NOTE: THESE ARE ALL REFERENCES TO THE BOOK
I this section I will introduce Finite Markov Decision Processes (MDPs) which serve as a formalization of sequential decision making. Problems which can be described as a finite MDP are what Reinforcement Learning is trying to solve. They are called finite because the sets $\mathset{S}, \mathset{A}, \mathset{R}$ of all states, actions, and rewards respectively, are all finite. I will illustrate and apply the concepts in this section using a Grid-World environment. It serves as a particularly convenient example since in it, the state of the environment is fully described by the agent's position. This allows for simple computation and visualization. 

\subsection{Sequential decision making}\label{subsec:sequential_decision_making}
In an MDP the agent and environment continually interact. At each time step $t$ of that interaction, the agent selects an action, the environment transitions in to a new state $S_{t+1}$ and gives the reward $R_{t+1}$. Based on this new state, the agent selects another action. This series of interactions may go on forever. From this there also follows that states can be revisited, a finite set of states could not support an infinite series of transitions otherwise. $S_ {t+1}$ is just the state which the environment transitions to next and does not refer to a specific state in the set of states, the environment may even transition back to itself, in that case $S_t$ and $S_{t+1}$ would be identical. This is possible because all possible futures solely depend on that state, else it would not fully describe the environment.

\begin{figure}[h!]
    \centering
    \includegraphics[width=0.7\linewidth]{figures/agent_environment_interaction_loop.png}
    \caption{Agent Environment Interaction Loop as presented in \citepg{48}}
    \label{fig:agent_env_inter}
\end{figure}

\noindent
Since the agent environment interaction is sequential, a sequence, or in Reinforcement Learning \textit{trajectory} $\tau$, arises. \citepg{48}

\begin{equation}\label{MDP:trajectory}
    \Tau = S_0, A_0, R_1, S_1, A_1, R_2 \dots
\end{equation}
\centerline{\small\textit{from \citepg{48}}}

\noindent
\\ In an \ita{episodic} environment, an environment that ends at some terminal time-step $T$, a trajectory has a terminal state $S_T$ and reward $R_T$. The notion of a terminal action does not make sense since the environment terminates at $T$, any action taken would not have an effect, and the action which proceeds $S_T$ is $A_{T-1}$.

\subsubsection{Illustrating in Grid-World}\label{subsubsec:grid_world_trajectory}
The Agent starts at a starting position, here the bottom left corner, and has to reach the upper right corner, by choosing from the actions up, down, left, right at each time step. The blacked out fields are inaccessible, running in to them or the walls, just results in the same state, the agent does not move. The rewards are not yet displayed here to avoid clutter, they will be discussed in the next section.

\begin{figure}[h!]
    \centering
    \includegraphics[width=0.7\linewidth]{figures/Grid_world_trajectory.jpeg}
    \caption{The Grid World Environment}
    \label{fig:grid_world}
\end{figure}

\noindent
The agent then moves through the states and therein generates a trajectory. The representation of that trajectory overlaid on to the grid, works in this case because the agents position fully describes an environment state. Archiving an agents trajectory becomes useful when evaluating and training the agent, since an evaluation based upon a series of correlated events, is obviously more meaningful, than one based entirely on one singular action and its consequence.

\subsection{Agent Policies in a MDP}\label{subsec:policies}

Moving forward it is important to understand what policies are mathematically. So far it has been established that policies are the way in which an chooses an action based on a state. Policies do not have to be deterministic. Mathematically, the policy is a function which takes a state and produces a probability distribution over the set of actions. This function can be parameterized, the parameter vecotr is commonly written as $\theta$. This distribution can be discrete or continuous and may produce vector actions. An action can be obtained from a policy by sampling from it. This is outlined in \citepg{58}, where $\pi(a|s)$ is defined as the probability of $a$ given $s$ under $\pi$. In our example of grid world the policy would be a discrete distribution. If the policy is parameterized, then $\pi(a|s)$ becomes $\pi(a|s,\theta)$. 

\subsection{The episodes Return, an improved measure of agent performance}\label{subsec:goals}

The agent selects actions based on state information and and receives rewards as feedback for its previous action. The goal of the agent is to maximize the reward signal received. It does that by updating its policy to produce a better action $A_t$ in the state $S_t$ based on the signal received at $t$. This means that one of the ways in which we can improve learning performance is to tweak the reward signal. The perhaps most obvious signal to give the agent is the reward $R_{t+1}$ which followed its action. However this does not take in to account any of the future states and rewards the agent's action lead to, it would not at all plan for any future states and rewards. To curb this issue we introduce the \textit{return} $G_t$. In contrast to the reward, the return also considers the influence of $A_t$ on future rewards. It is defined as follows:

\begin{equation}\label{MDP:return}
    G_t \doteq R_t + R_{t+1} + R_{t+2} \dots + R_T
\end{equation}
\centerline{\small{\ita{\citepg{54}}}}

\noindent
\\ Although the return undoubtedly is a better measure of the "goodness" of an action than the reward and training the agent using it produces decent results, it still has some flaws. Of these the most relevant to my work is that taking the sum of the rewards from the current time step to the episode's end gives the same weight to all the rewards. In many environments $A_{t-600}$ is much less relevant to $R_{t+1}$ then $A_t$, yet $R_{t+1}$ has the same weight in $G_{t}$ as it has in $G_{t-600}$. To address this, the discounted return can be used. 

\begin{equation}\label{MDP:discounted_return}
    G_t = \sum_{k=t}^{T} R_{t+k+1} \cdot \gamma ^k 
\end{equation}
or recursively as
\begin{equation}\label{MDP:recursive_discounted_return}
    R_{t+1} + \gamma \cdot G_{t+2} \mathrm{\ where\ } 0 \leq \gamma \leq 1
\end{equation}
\centerline{\small{\ita{\citepg{55}}}}

\noindent
\\ By tweaking the discount factor we can balance immediate reward with future ones. It is another hyper parameter that needs to be set which affects agent performance. Discounting also enables the use of the return in continuous environments, it is why it was originally introduced and is the main reason it is presented in \cite{sutton_reinforcement_2018}. {\\ \color{red} SHOW RESULTS OF DISCOUNT FACTOR 1 VS 0.99 AND SAY THAT IT IS BAD} \\ The returns in Grid-World with different discount factors can be seen here:

\begin{figure}[h!]
    \centering
    \includegraphics[width=\linewidth]{figures/grid_world_discount_factors.png}
    \caption{Returns for each time step in an episode for different discount factors}
    \label{fig:grid_world_discount_factors}
\end{figure}

\noindent
\\ The only reward the agent receives here occurs on the final time-step, and has a magnitude of 1. Notice how quickly the return decays with even a discount factor 0.9. This makes the agent quite short sighted. Due to its exponential decay small discount factors often do not make sense.

\noindent
\\ Optimizing for the return is still dependant on the quality of the rewards. Ideally, rewards should represent the thing we want the agent to achieve. Since they are the only information the agent can learn from should also not be too sparse. The problem of dealing with environments which give sparse reward is a large hurdle in modern day RL, some solutions to this are addressed in \citepg{491}. In such environments it turns out to be useful to modify the rewards received by setting baselines, creating "intrinsic rewards", those could for example be rewards for exploring the environment and many others. 

\subsection{Simple Solution for Grid-World using the presented concepts}
TODO



\section{Optimization with Gradient Descent}\label{gd}
All throughout Machine Learning gradient descent and variations of it are used for optimization. Gradient descent is vital for my work. As such I will briefly explain what it is and how it is relevant to my work.

\subsection{What is Gradient Descent?}\label{gd:what_is_it}
In essence gradient descent is an algorithm used to find local minima of differentiable functions. It can be thought of as starting at a point on the function graph and walking down hill until one reaches a minimum. Gradient descent works by differentiating a function at some point in order to obtain a slope, and then moving in the downwards direction of that slope. Gradient descent is done in steps. If $f(\theta)$ is a differentiable function, then one gradient descent step is:

\begin{equation}\label{Graident_Descent:basic_update}
    \theta \leftarrow \theta - \alpha \cdot \frac{\partial}{\partial \theta}f(\theta)
\end{equation}

\noindent
\\ The \textit{step size} $\alpha$ is introduced to control the magnitude of the descent step. The gradient descent algorithm consists of performing these steps iteratively until the current value (in the above case $\theta$) is sufficiently close to the actual local minimum. Performing 10 gradient descent steps on the function $f(\theta) = \theta^2$ and a starting value of $\theta = 3$ with various values for alpha looks as follows:

\begin{figure}[ht]
    \centering
    \includegraphics[width=\linewidth]{figures/grid_world_discount_factors.png}
    \caption{Gradient Descent on $f(\theta) = \theta^2$}
    \label{fig:UNO}
\end{figure}

\noindent
\\ As can be clearly seen here smaller step sizes lead to more accurate results. Too large a step size may even lead to divergence. However, if the step size is too small, it may take very long to reach a local minimum. Gradient descent functions in the exact same way if $\theta$ is a vector. In that case the update step becomes:

\begin{equation}\label{Graident_Descent:basic_update_vector}
    \theta \leftarrow \theta - \alpha \cdot \nabla_\theta f(\theta)
\end{equation}

\subsubsection{Gradient Descent on functions which take unknown variables}\label{gd:random_var}
Gradient Descent can also be used to minimize function parameters when unknown variables are in the mix, and the goal is to find parameters which lead to minimal function values for most parameters. Say $f(x, \theta)$ is the function we are performing gradient descent on, $\theta$ is a vector of parameters we are trying to optimize, and $x$ is sampled from a distribution. We can perform gradient descent steps which are close to optimal, by calculating the gradients $\nabla_\theta f(x, \theta)$ for $n$ number of samples of $x$, averaging them, and performing a gradient descent step on that average. The more samples of $x$ we have the more accurate the direction and magnitude of the step are going to be going to be.

\subsubsection*{Gradient Ascent}\label{gd:gradient_ascent}
As the name implies gradient ascent is the exact opposite of gradient descent. Instead of trying to find the local minimum of a function, when performing gradient ascent we are trying to find the local maximum of a function. Gradient Ascent is identical to performing Gradient Descent on the negative derivatives. 

\subsection{How is Gradient Descent relevant to Reinforcement Learning?}\label{gd:relevance}
In the previous section we have established that policies are functions which take actions and produce probability distributions over a set of actions. They may also be parameterized. One very common way of optimizing policies is to define a function $J(\theta)$, which measures agent performance \citepg{312}. This function is dependant on, and differentiable with respect to, the policy parameters $\theta$. Gradient Descent or Ascent (which one we use is dependant on if the performance measuring function treats lower or larger values as better) can then be used to optimize $\theta$. I will present one such function, in section \ref{sec:policy_gradient}. Gathering a set of agent-environment interactions, using the rewards received in them in a performance measuring function which depends on the policy parameters, and then using a variation on Gradient Descent to optimize those parameters, is the method i use to train an agent in this thesis. This can be represented in an abstract algorithmic fashion:

\begin{algorithm}[H]
\SetAlgoLined
 initialization\;
 \Repeat{agent performance suffices}{
  \For{$t \in 0, \dots, n$}{
   Get state $S_t$\;
   Take action $A_t$ sampled from $\pi(a|s,\theta)$\;
   Receive reward $R_t$\;
  }
  Compute gradients of performance function based on rewards w.r.t. $\theta$\;
  update $\theta$ with Gradient Descent;
  
 }
 \caption{Reinforcement Learning With Gradient Descent}
\end{algorithm}

% !TEX root = ../maturaarbeit.tex
\newpage
\section{Solutions with Policy Gradient Methods: WORK IN PROGRESS}\label{sec:policy_gradient}
\noindent
In this section I shall briefly present Policy Gradient methods, their advantages and aim to give an intuitive understanding of their operation to the reader. Policy Gradients (PG) are the basis of the algorithm used in this thesis to solve the soccer problem. They are one of the main methods used in modern Reinforcement Learning. Countless algorithms build on them, one of which I will present and use here. The reason I chose policy gradient methods for my work their ability to elegantly handle continuous action spaces as well as their high relevance in state of the art Reinforcement Learning. However, there are other widely used and highly viable methods as well \cite{arulkumaran_brief_2017}. Policy Gradient Methods Rely on the concept of the policy as a function which produces a probability distribution over the action space, and the concept of gradient ascent, which i discussed in the previous section \ref{gd:gradient_ascent}. At the end of that section I mentioned how Gradient Descent / Ascent could be used in Learning, By defining a Performance Measure dependant on the policy's parameters, and then differentiating that with respect to those parameters, thereby finding out in what direction to change them in order to improve the policy. Here I will present just such a measure, denoted as $J(\theta)$. 

\subsection{The Policy Gradient Theorem}\label{subsec:pg:theorem}
The policy gradient theorem provides the derivative of $J(\theta)$ w.rt. $\theta$. Conceptually $J(\theta)$ is the true value of of $\pi$ with the parameters $\theta$ starting in a state $s_0$. It can be thought of as a measure of how well an agent following the policy will perform starting at $s_0$. In \cite{sutton_reinforcement_2018} $J(\theta)$ it is defined as follows:

\begin{equation}\label{pg:performance_measure}
    J(\theta) \doteq \mathbb{E}_{\pi_\theta}[G_t|S_t=s_0]
\end{equation}

\noindent
In \citepg{324} $j(\theta)$ is technically defined to be $v_{\pi_\theta}(s_0)$ \citepg{58}, which is defined to be the above earlier in the book. Since I do not make use of $v_{\pi_\theta}(s_0)$, I am going to equate the two here, the differences are conceptual. The proof as taken from \citepg{325} can be found in appendix \ref{appendix:sec:pg}. It requires some additional concepts, the introduction of which extend past the scope of this thesis. But if you are interested, for a better understanding I suggest you read through the early chapters of \cite{sutton_reinforcement_2018}. This can then be simplified:
\begin{align}
    \nabla J(\theta) &\propto \sum_{s} \mu(s)\sum_a q_\pi(s,a) \nabla \pi(a|s,\theta) \nonumber \MoveEqLeft[1] \\ 
    &= \mathbb{E}_\pi \left[\sum_a q_\pi(S_t,a) \nabla \pi(a|S_t,\theta)\right] \nonumber \\
    &= \mathbb{E}_\pi \left[\sum_a \pi(a|S_t,\theta) q_\pi(S_t,a) \frac{\nabla \pi(a|S_t,\theta)}{\pi(a|S_t,\theta)}\right] \nonumber \\ 
    &= \mathbb{E}_\pi \left[q_\pi(S_t,A_t) \frac{\nabla \pi(A_t|S_t,\theta)}{\pi(A_t|S_t,\theta)}\right] \nonumber \tag*{(replacing $a$ by the sample $A \sim \pi$)}\\
    &= \mathbb{E}_\pi \left[G_t \frac{\nabla \pi(A_t|S_t,\theta)}{\pi(A_t|S_t,\theta)}\right] \tag*{(because $\mathbb{E}_\pi \left[ G_t|S_t, A_t \right] = q_\pi \left(S_t, A_t \right)$)}
\end{align}

\centerline{From \citepg{326, 327}}

\noindent
\\ \\ Consider what this actually means. This gradient tells us what direction to nudge $\theta$ in in order to maximize $J(\theta)$. The optimal direction to nudge $\theta$ in will be different for every sample interaction with the environment. The actual best direction is determined by the combined nudges from all samples. That \textit{is} the expectation of the gradient when sampled from interactions of an agent following $\pi$ , with the environment.
\nolinebreak
The next term, the gradient of the the probability, $\nabla \pi(A_t|S_t, \theta)$, of action $A_t$ being picked in state $S_t$, multiplied by $G_t$, thereby being proportional to it, can be though of as the direction in which to move $\theta$ in order for $a$ to be taken more often, scaled by the "goodness" of $A_t$. When using this gradient, action of high value are going to be made more probable to be taken than action of low value. This even holds true when the return is always positive or always negative, because the sum of the probability of all actions must be $1$. The probability of one action is invariably linked to the probability of all others, when all probabilities are increased, the highest ones will simply be those which were increased the most. This also leads to the last component of this gradient. The division by $\pi(A_t|S_t, \theta)$, the probability of the action itself. This makes sense because, quoting from \citepg{327}, \textit{"otherwise actions that are selected frequently are at an advantage (the updates will be more often in their direction) and might win out even if they do not yield the highest return." }

\subsection{Policy Gradient Based Learning with REINFORCE}\label{subsec:pg:reinforce}
This is the gradient vecotor of $J(\theta)$ with respect to $\theta$, and what I use during gradient descent. 

% !TEX root = ../maturaarbeit.tex
\newpage
\section{Function approximation using Artificial Neural Networks}\label{sec:neural_networks}
All throughout this chapter I have talked about the policy being a function which maps environment states to a probability distribution over the set of actions. It follows that an optimal policy $\pi^*(a|s)$ is a function which maps the respective best possible distribution to each state. In reinforcement learning we do not know that optimal policy, if we did there wouldn't be any problem to solve. \textit{Artificial Neural Networks} (ANN) offer a method of approximating a general target function $f^*(x)$ \cite[p. 164]{Goodfellow-et-al-2016}. In policy gradient based reinforcement learning they are commonly used to directly approximate the optimal policy. In this section I will cover how a simple type of ANN, a \textit{multilayer perceptron} (MLP), functions. To explain MLPs I will first cover what a perceptron is, and then explain how multiple perceptrons can be combined to form an MLP.

\subsubsection{Perceptron}\label{subsubsec:nn:comp:perceptron}
As their name implies, artificial neural networks are made up of neurons, or rather artificial ones. One type of such is neuron is the perceptron \cite[chap. 1]{nielsen_neural_2015}. A perceptron, takes several weighted inputs and produces a single output. Unlike in other types of artificial neurons, in a perceptron that output is binary. 

\begin{figure}[H]
    \centering
    \includegraphics[width=0.6\linewidth]{figures/placeholder.png}
    \caption{Caption}
    \label{fig:my_label}
\end{figure}
\noindent
This figure illustrates how in a perceptron the weighted inputs are summed, this sum is denoted as $z$ here, and then passed to a step function, here $a(z)$, which produces the perceptions final binary output. Notice how $z$ is essentially the dot product between an input vector $\mathbf{x} = \{x_0, x_1, \dots, x_j\}$ and a weight vector $\mathbf{w} = \{ w_0, w_1, \dots, w_j \}$. Using this the perceptrons output becomes $a(\mathbf{x} \cdot \mathbf{w})$. Conceptually a perceptron can make a yes / no decision based upon some arbitrary number of weighted inputs. Consequentially, a perceptrons weights encode its behaviour. 

\subsubsection{Layers}\label{subsubsec:nn:comp:layers}
A single perceptron is not of much use when approximating some general function, let alone a policy. To expand upon their capability perceptrons can be combined in to layers. A layer of perceptrons still takes an input vector $\mathbf{x}$, which all perceptrons in the layer are connected to. Unlike a single perceptron however, a layers output is itself a vector, namely the vector of the outputs of all it's perceptrons.

\begin{figure}[H]
    \centering
    \includegraphics[width=0.6\linewidth]{figures/placeholder.png}
    \caption{Caption}
    \label{fig:my_label_1}
\end{figure}
\noindent
To compute the output of this vector for a given input, $z_i$ of each perceptron $i$ would need to be computed, and then passed to the step function $a(z)$. However, since $z_i$ is just the dot-product $\mathbf{x} \cdot \mathbf{w}_i$, a weight matrix $\mathbf{W}_{i, j} = \left[ \mathbf{w}_0, \mathbf{w}_1, \dots, \mathbf{w}_i \right]$ applied to $\mathbf{x}$ can be used instead, to get the vector $\mathbf{z} = \left\{ z_0, z_1, \dots, z_i \right\}$. To this $a(z)$ can then be applied element wise, to get the vector of outputs of the layer. I will use $a(\mathbf{z}_i)$ as shorthand for this element wise application. The output of a layer $f$ is:

\begin{equation}
    f(x, \theta) = a \left( \mathbf{W} \cdot \mathbf{x} \right)
\end{equation}
\noindent
Layers can then be composited to form an artificial neural network, specifically a multilayer perceptron. In \cite[p. 164]{Goodfellow-et-al-2016} the notation used for this is $f(x) = f^{(n)}(f^{(\dots)}(f^{(2)}(f^{(1)}(x))))$. Here $f$ is the entire MLP, and $f^{(n)}$ are all its layers. Of course $f(x)$ is not only dependant on $x$ but on all the weights in the layers weight matrices. As per the notation used in \cite{Goodfellow-et-al-2016} I will use $\theta$ to encompass all these function parameters, thereby $f(x, \theta)$ denotes the network. 

\subsubsection{Activation Functions and Biases}\label{subsubsec:nn:comp:activation}
Perceptrons use a step function to produce a binary output. However, in modern neural networks a variety of other so called \textit{activation functions} may be used. Common ones include the logistic function, the arctangent and rectified linear activations like $\max(x, 0)$, which I primarily use for my work. The activation function of a neuron must not be linear. This is because then the network would just be comprised of a series of linear functions, which could be written as a single one, thereby making multiple layers redundant. this would significantly limits the information which can be encoded by the network. 
\noindent
\\ The final missing piece in an MLP are the biases. Biases act to shift the input of a neuron to its activation function by a input independent baseline value, thereby also shifting its output, effectively biasing it. By slight modification of a perceptron we get the equation for a more general neuron in an MLP:

\begin{equation}\label{neuron}
    f(x) = a(b + \sum_i w_i \cdot x_i)
\end{equation}

\noindent
A rearranged example of this can be found in \cite[chap. 1]{nielsen_neural_2015} in the section on sigmoid neurons (the sigmoid function used there is the logistic function, they are often used interchangeably). In notation, biases are also included in $\theta$.

\subsection{Solving Grid World with Policy Approximation PROLLY NOT SIKE}\label{subsec:nn:example}
In this section i am going to use a simple MLP to approximate an optimal policy for playing slot machines. The exact environmen
% !TEX root = ../maturaarbeit.tex
\chapter{Implementation Details}\label{chap:in_practice}
In this chapter I will apply Reinforcement Learning to a simplified simulation of soccer. I will go over the environment specification, the agent implementation, training results and what was required to get there.

\section{Frameworks and Tools}\label{sec:ip:tools}
In this section I will briefly go over the main tools I used for my work and provide reasons for why I opted to use them.
\subsection{Python}\label{subsec:ip:tools:python}
%Machine Learning Libraries
I used Python as the main language to write my agent code in mainly because of all the data science and machine learning libraries present for it. Tools like NumPy \cite{noauthor_numpy_nodate}, TensorFlow all provide highly performant and parallel backends for data storage, manipulation and vectorized operations. This also nullifies the relatively bad performance of python, very few operations are actually performed by it.
\nolinebreak
%Low iteration times
Another major plus for python is the low iteration time when developing for it. Because it is interpreted rather than compiled there is no waiting time for code to compile and live python environments like IPython \cite{noauthor_ipython_nodate} and Jupyter Notebook \cite{noauthor_jupyter_nodate} enable excellent testing.
\nolinebreak
%Modularity
Last but not least there is the excellent standard library of python. It makes asynchronous programming, file handling efficient storage through Deques and general boiler plate operations easy.

\subsection{TensorFlow 2}\label{subsec:ip:tools:tensorflow}
%what
TensorFlow is a Machine Learning API \cite{noauthor_tensorflow_nodate}. I use it due to my prior familiarity with it and its broad set of features. It is widely adopted and enables parallel execution of vectorized operation on GPUs through CUDA and cuDNN. Furthermore it also provides tools for FIFO-queues and data visualization through TensorBoard. TensorFlow 2.X allows for the use of compiled computational graphs, as well in place "eager" execution of operations. It has provisions for automatic differentiation and through its Keras module makes building Artificial Neural Networks simple.

\subsection{Unity ML-Agents}\label{subsec:ip:tools:ml_agents}
%what
I use the Unity ML-Agents toolkit \cite{juliani2020unity} \cite{noauthor_unity-technologiesml-agents_2020} for my environment implementation. It provides example environments under the Apache 2.0 license, I use a modified version of one of them. 
\nolinebreak
%why
I opted to use this toolkit because of my prior knowledge with the Unity game engine it is based in, its flexibility in creating and modifying environments, it's visual appeal, the possibility of easily implementing human input in order to play against trained agents (where such play applies) and the high quality environments which come prepackaged.


\section{The Environment}\label{sec:ip:environment}
\begin{figure}[H]
    \centering
    \includegraphics[width=0.7\linewidth]{figures/soccer_field.png}
    \caption{A screenshot of the soccer field taken from the Unity editor}
    \label{fig:soccer_field}
\end{figure}

\noindent
%Overview with image
Here I will lay out the workings of the environment I opted to use, explain the integration with with python, and go over some of the code crucial to its function. This section describes the baseline  modified "Soccer" environment. "Soccer" is the name used in the ML-Agents toolkit. I work with variations on it in \ref{sec:tr:variations}, they all depend on this baseline. The environment I use to train the agent is a simplified version of a game of soccer. 

\subsection{Suitability for my Work}\label{subsec:ip:environment:suitabilty}
At the beginning of this thesis I was faced with the decision whether I should use a preexisting environment or create my own. I opted to make use of an already existing one since this was more in line with my lead question, which asks how an agent is trained. I chose this specific environment because it presents a decent balance between complexity, likelyhood of successful training, intrigue and extensabilty. Due to its extensabilty i could also train and test agents on variations of it, thereby having a measure of how adaptable it is. Further after the modifications I made to the environment it now also offers continuous control to the agent, this presents an interesting challenge, relevant to real world task, where input is seldom binary.

\subsection{Specifics of Operation}\label{subsec:ip:environment:implementation}
The baseline environment is in function mostly identical to the "Soccer" example environment provided by the Unity ML-Agents toolkit. Here I will go over it's sequence of operation, it's specification, and state all the changes I made to it in order to make it suit my work better. The game of soccer takes place on a bounded field and has four players, distributed across two teams. All Each Player is its own Agent, they all act on the same policy and all produce trajectories for it's training. Thereby the policy is trained in play against itself. 

\newpage

\subsubsection{Observations}\label{subsubsec:ip:environment:impl:observations}

\begin{figure}[H]
    \centering
    \includegraphics[width=1\linewidth]{figures/ray_sensor.png}
    \caption{Caption}
    \label{fig:ray_sensor}
\end{figure}
\noindent
The "Soccer" example environment uses "Ray Perception Sensors" to create an observation for the agent. A ray perception sensor works by casting a number of rays spread out through a predefined field of view. Along each of these rays a sphere is slid, starting at the agents position and stopping on collision with an object. Each ray contains information about the distance to the object it hit, and its type. The types are predefined, in the case of "Soccer" they are "wall", "enemy", "teammate", "team goal", "enemy goal" and "ball". To the agent a vector is then passed, consisting of the distance to the object the ray struck, as well as its one-hot encoded type. The above figure illustrates how one-hot encoding works and how the observation vectors looks for each ray. The observation vectors are then concatenated to form one single vector, this is then used as the input layer of my policy network. One-hot encoding is commonly used in machine learning where a type information needs to be passed as input, this way separate trainable weights are present for each category. The episode ends if either a goal is score, or a maximum of $600$ time steps pass.

\subsubsection{Episode Start}\label{subsubsec:ip:environment:impl:start}
\textcolor{red}{\textbf{WRITE IN CHAPTER \ref{chap:training} ABOUT THE RANDOMIZATION}}

\subsubsection{Actions}\label{subsubsec:ip:environment:impl:actions}
The environment takes continuous action input from the agent. This was not the case in the example environment provided. Three single precision floating point values are expected, they represent forward, lateral, and rotational movement respectively. This is the \code{C\#} code I use to move the agent:

\begin{lstlisting}[basicstyle=\footnotesize]
public void MoveAgent(ActionSegment<float> act) {
    m_KickPower = 0f;

    float longitudinalAxis = Convert.ToSingle(System.Math.Tanh(act[0])) * m_ForwardSpeed;
    float lateralAxis = Convert.ToSingle(System.Math.Tanh(act[1])) * m_LateralSpeed;
    float rotationalAxis = Convert.ToSingle(System.Math.Tanh(act[2]));

    Transform transformTMP;
    (transformTMP = transform).Rotate(transform.up, rotationalAxis * 6.5f);
    agentRb.AddForce(
        (transformTMP.forward * longitudinalAxis +
         transformTMP.right * lateralAxis)
        * m_SoccerSettings.agentRunSpeed,
        ForceMode.VelocityChange);
}
\end{lstlisting}
\noindent
As visible here, I use the scaled arctangent of the action values received. This is to prevent unexpected behaviour which might result when the values the agent selects are too large. The scaling values for forward and lateral speed are the same as the ones used in the "Soccer" example environment. They are \code{m\_LateralSpeed \= 0.3f;} and \code{m\_ForwardSpeed \= 1.0f;}. The \code{agentRunSpeed} is \code{2.0f}.

%Continuous Action Space
%tanh to bind 
\subsubsection{Rewards}\label{subsubsec:ip:environment:impl:rewards}
In the baseline environment I use the rewards provided by the ML-Agents toolkit. The only rewards given to the agent over the course of the episode are on collision with the ball, in which case a reward of \code{0.2} is given, and the end of the episode. This reward is \code{1.0 + timePenalty} if the episode terminated because the agents team managed to score a goal, if the episode ended in a draw or the agent's team lost it is \code{-1.0}. The \code{timePenalty} is accumulated at every time step according to \code{timePenalty -= m_Existential;}, with \code{m_Existential} being the inverse of the maximum episode length. If there was no time penalty it might me more advantageous for the agents to only repeatedly hit the ball instead of scoring goals.
%perform same action for 10 steps
%quit after n steps
%random agent heurisitc 
%Some Code Snippets 
\subsection{Communication with Policy in Python}\label{subsec:ip:environment:communication_python}
One of the reasons the ML-Agents toolkit is particularly appealing is because it allows for extensive control of the environment in Python. The toolkit provides a Python side API with the \code{mlagents} and \code{mlagents_envs} modules. The figure below, taken from \cite[p. 12]{juliani2020unity}, illustrates this interface.

\begin{figure}[H]
    \centering
    \includegraphics[width=0.75\linewidth]{figures/ml_agents_python_communicator.png}
    \caption{Caption}
    \label{fig:python_communicator}
\end{figure}

\noindent
Through the Python API it is possible to train multiple agents, with possibly differing Behaviours (formally, each of these behaviours would represent a different environment because the observations and actions, as well as the underlying environment functions differ from those of other behaviours, they are effectively a different MDP, see section \ref{sec:MDP}), and even to have yet other agents running on inference (acting on policies which are not being updated, here "Neural Network") or heuristics. This allows for tremendous flexibility not only in training but also allowed me to more fully consider lead question which also pertains to observation, action and reward design. On the Python side of the ML-Agents training framework there exists aside from various other classes which I will briefly cover in section \ref{sec:ip:agent_implementation}, the \code{UnityEnvironment} class. Instantiating it launches, and creates an interface with a Unity executable \cite[p. 13]{juliani2020unity}. From an instance of \code{UnityEnvironment} information about the observation and action type and size can be obtained. During training and evaluation observations, actions, stepping of the environment, and restarting episodes are all handled through this class's members.

\section{The Agent Implementation}\label{sec:ip:agent_implementation}
\subsection{The Agent Policy Network}\label{subsec:ip:agent:network}
%Some Network Code
%Structure 
%Why I chose that structure
\subsection{The Parameter Update Function}\label{subsec:ip:agent:training_func}
%S
\subsection{Efficient Computation of Returns}\label{subsec:ip:agnet:return_calc}
\subsection{Caching experience, delayed Training}\label{subsec:ip:agent:archive}
\section{End-to-End Training and Evaluation}\label{sec:ip:training_testing}
\subsection{The Training Loop}\label{subsec:ip:tt:training_loop}
\subsection{Evaluation}\label{subsec:ip:tt:eval}


% !TEX root = ../maturaarbeit.tex
\chapter{Training the Agent}\label{chap:training}
\section{Baseline Results}\label{sec:tr:baseline_results}
\section{Lower Iteration Times through Parallelization}\label{sec:tr:parallel}
\section{Parameter Tweaking}\label{sec:tr:param_tweaking}
\section{Variations on the Environment}\label{sec:tr:variations}

\chapter{Discussion}\label{chap:discussion}
\section{Validity of Data}\label{sec:disc:significance}
The Error in win percentages between parameters variations is fairly low, games between agents are technically multinomial distributions but because ties are so infrequent (on average $\approx$ 0.5\% of outcomes), and all my data for win probabilities is close to 0.5, i use the "Wade" confidence interval of a binomial\cite[p. 2]{binomial_confidence} to approximate it. I accept a high error level\footnote{$1-\alpha$ can be thought of as the confidence that data falling outside the interval obtained with it, is meaningful.} of $\alpha = 0.2$, as this thesis' focus lies not in perfect parameters, but methods for agent training. The error obtained through this method turns out to be $\approx$0.013\footnote{$p \pm e, e = z\sqrt{p(1-p)/n}$ z is the number of standard deviations within which the percentage of data specified by $\alpha$ falls \cite[p. 2]{binomial_confidence}. It can be obtained by solving $1-\alpha = 2\int_0^z \sqrt{2\pi}^{-1}\exp(-t^2/2)dt$ for $z$.} which corresponds to 1.3\% for all parameters. This corresponds to a difference in elo between parameters of approximately 9\footnote{$\Delta elo = \log_{10}\left(\left(p_{win}^{-1}-1\right)^{-400}\right)$, this increases by $\approx$ 9.0 $\pm$ 0.5 for a percent error of 1 if $p_{win} = 0.5$ $\pm$ 0.12 which all of my data falls in to.} points. The more troublesome error is in the difference between between runs. I cannot control the random seed used for Unity, which makes 1 to 1 repeatability of runs impossible. Monte Carlo approximation of actual parameter performance is prohibitively expensive, instead I estimate the variance in runs as

\begin{equation*}
    \sigma^2 = \frac{1}{n-1} \sum_{i=1}^{n} \left( x_i - \overline{x} \right)^2
\end{equation*}

\noindent
and add to the variance in the binomial confidence interval. The variance of 8 sample runs at 10 million steps is $\approx$ 0.00035, corresponding to an error of $0.028 = 1.3\cdot \sqrt{p(1-p)/2560 + 0.00035}$ if $p_{win} = 0.5 \pm 0.12$, this holds for all my data. however this again imperfect as all data is interdependant. This can be mitigated by letting all sample agent's play against a common opponent, and using the average win percentage of the results for the variance, this methods yields: $\sigma^2 \approx 0.00010$, leading to an error of 0.018, again with $p_{win}$ = 0.5. This is perhaps more representative data as it more accurately reflects the scenario of comparison between parameters, it is the value which I will use, it corresponds to a $\Delta elo$ between parameters of $\approx 13.4$. 
\\Further, 20 million time steps might not be fully representative of agent performance after 100 million. For me, the computational cost of full training runs is simply too high. The elo rating system is a decent indicator of overall performance, but might not give the best parameter choice when testing.

\section{Training Process}
 Agent training for even a relatively simple problem such as this soccer simulation incurs immense computational and temporal cost, including initial testing, though much of this was in parallel, total training time during this thesis including initial testing was roughly 250 hours. Some of this was due to inefficiencies in my methodology, and perhaps poor code performance. As the training algorithm gets more complex and additional variables are introduced, this issue only worsens. 
 \\ Evaluation of training data is also quite difficult, especially in tasks where self play is involved. The elo rating system is a decent indicator of overall performance, but might not give the best parameter choice when testing.
\section{Results}

\section{Outlook}
% !TEX root = ../maturaarbeit.tex
\chapter{Conclusion}\label{chap:conclusion}
% talk about process
% talk about doing actuall rl not restrict
% throw some shade
%say throw away shit network data


% frage beantworten EXPLIZIT

% was gelernt

% bezug auf diskussion was anders machen
% !TEX root = ../maturaarbeit.tex
\printbibliography[
heading=bibintoc,
title={Bibliography}
]

\appendix
\listoffigures 
\listoftables 
% !TEX root = ../maturaarbeit.tex

% !TEX root = ../maturaarbeit.tex

\chapter{Proofs}\label{appendix:proofs}
\section{Policy Gradient Theorem}\label{appendix:sec:pg}
\begin{tcolorbox}[colback=white!5!white,colframe=white!50!black,
  colbacktitle=white!75!black,title=Proof of the Policy Gradient Theorem (episodic case)]
  For the following proof consider $q_\pi(s,a)$ to be defined as $\mathbb{E}_\pi[G_t|S_t=s,A_t=a]$ \citepg{58}, and $\pi$ to be a function of $\theta$.
  \tcblower
  \begin{align*}
  \nabla v_\pi(s) &= \nabla \left[\sum_{a}\pi(a|s)q_\pi(s,a)\right] \MoveEqLeft[1]\\
  &= \sum_{a}\left[\nabla \pi(a|s)q_\pi(s,a)+\pi(a|s)\nabla q_\pi(s,a)\right] \\
  &= \sum \left[\nabla \pi(a|s)q_\pi(s,a) + \pi(a|s)\nabla \sum_{s', r} p(s',r|s,a) (r + v_\pi(s'))\right] \\
  &= \sum \left[\nabla \pi(a|s)q_\pi(s,a) + \pi(a|s) \sum_{s'} p(s'|s,a) v_\pi(s')\right] \\
  &= \sum_{a}\left[\nabla \pi(a|s)q_\pi(a,s) + \pi(a|s)\sum_{s'}p(s'|s,a) \right. \\
  &\qquad \left. \sum_{a'}[\nabla \pi(a'|s')q_\pi(a',s') + \pi(a'|s')\sum_{s''}p(s''|s',a')\nabla v_\pi(s'')]\right] \\
  &= \sum_{x \in \mathset{S}}\sum_{k=0}^{\infty}\operatorname{Pr}(s \rightarrow x, k, \pi)\sum_a \nabla \pi(a|x)q_\pi(x,a), \\
  \intertext{
  after repeated unrolling, where $\operatorname{Pr}(s \rightarrow x, k, \pi)$ is the probability of transitioning from state $s$ to state $x$ in $k$ steps under the policy $\pi$. It is then immediate that 
  } 
  \nabla J(\theta) &= \nabla v_\pi(s_0) \\
  &= \sum_{s}\left(\sum_{k=0}^{\infty} \operatorname{Pr}( s_0 \rightarrow s, k, \pi)\right)\sum_{a}\nabla \pi(a|s)q_\pi(s,a) \\
  &= \sum_s \eta(s) \sum_{a}\nabla \pi(a|s)q_\pi(s,a) \tag*{\citepg{199}} \\
  &= \sum_{s'} \eta(s') \sum_s \frac{\eta(s)}{\sum_{s'}\eta(s')} \sum_{a}\nabla \pi(a|s)q_\pi(s,a) \\
  &= \sum_{s'} \eta(s') \sum_s \mu(s) \sum_{a}\nabla \pi(a|s)q_\pi(s,a) \tag*{\citepg{199}}\\
  &\propto \sum_s \mu(s) \sum_{a}\nabla \pi(a|s)q_\pi(s,a) \tag*{\scshape q.\:e.\:d.}
  \end{align*}
 
\end{tcolorbox}

% !TEX root = ../maturaarbeit.tex

\chapter{Code}\label{appendix:code}
\section{Move Agent Method}\label{appendix:code:move_agent}
\begin{lstlisting}[basicstyle=\footnotesize]
public void MoveAgent(ActionSegment<float> act) {
    m_KickPower = 0f;

    float longitudinalAxis = Convert.ToSingle(System.Math.Tanh(act[0])) * m_ForwardSpeed;
    float lateralAxis = Convert.ToSingle(System.Math.Tanh(act[1])) * m_LateralSpeed;
    float rotationalAxis = Convert.ToSingle(System.Math.Tanh(act[2]));

    Transform transformTMP;
    (transformTMP = transform).Rotate(transform.up, rotationalAxis * 6.5f);
    agentRb.AddForce(
        (transformTMP.forward * longitudinalAxis +
         transformTMP.right * lateralAxis)
        * m_SoccerSettings.agentRunSpeed,
        ForceMode.VelocityChange);
}
\end{lstlisting}
\end{document}